% Options for packages loaded elsewhere
\PassOptionsToPackage{unicode}{hyperref}
\PassOptionsToPackage{hyphens}{url}
\PassOptionsToPackage{dvipsnames,svgnames,x11names}{xcolor}
%
\documentclass[
  11pt,
  a4paper,
]{article}

\usepackage{amsmath,amssymb}
\usepackage{setspace}
\usepackage{iftex}
\ifPDFTeX
  \usepackage[T1]{fontenc}
  \usepackage[utf8]{inputenc}
  \usepackage{textcomp} % provide euro and other symbols
\else % if luatex or xetex
  \usepackage{unicode-math}
  \defaultfontfeatures{Scale=MatchLowercase}
  \defaultfontfeatures[\rmfamily]{Ligatures=TeX,Scale=1}
\fi
\usepackage{lmodern}
\ifPDFTeX\else  
    % xetex/luatex font selection
\fi
% Use upquote if available, for straight quotes in verbatim environments
\IfFileExists{upquote.sty}{\usepackage{upquote}}{}
\IfFileExists{microtype.sty}{% use microtype if available
  \usepackage[]{microtype}
  \UseMicrotypeSet[protrusion]{basicmath} % disable protrusion for tt fonts
}{}
\makeatletter
\@ifundefined{KOMAClassName}{% if non-KOMA class
  \IfFileExists{parskip.sty}{%
    \usepackage{parskip}
  }{% else
    \setlength{\parindent}{0pt}
    \setlength{\parskip}{6pt plus 2pt minus 1pt}}
}{% if KOMA class
  \KOMAoptions{parskip=half}}
\makeatother
\usepackage{xcolor}
\usepackage[top=2.5cm,bottom=2.5cm,left=2cm,right=2cm]{geometry}
\setlength{\emergencystretch}{3em} % prevent overfull lines
\setcounter{secnumdepth}{-\maxdimen} % remove section numbering


\providecommand{\tightlist}{%
  \setlength{\itemsep}{0pt}\setlength{\parskip}{0pt}}\usepackage{longtable,booktabs,array}
\usepackage{calc} % for calculating minipage widths
% Correct order of tables after \paragraph or \subparagraph
\usepackage{etoolbox}
\makeatletter
\patchcmd\longtable{\par}{\if@noskipsec\mbox{}\fi\par}{}{}
\makeatother
% Allow footnotes in longtable head/foot
\IfFileExists{footnotehyper.sty}{\usepackage{footnotehyper}}{\usepackage{footnote}}
\makesavenoteenv{longtable}
\usepackage{graphicx}
\makeatletter
\newsavebox\pandoc@box
\newcommand*\pandocbounded[1]{% scales image to fit in text height/width
  \sbox\pandoc@box{#1}%
  \Gscale@div\@tempa{\textheight}{\dimexpr\ht\pandoc@box+\dp\pandoc@box\relax}%
  \Gscale@div\@tempb{\linewidth}{\wd\pandoc@box}%
  \ifdim\@tempb\p@<\@tempa\p@\let\@tempa\@tempb\fi% select the smaller of both
  \ifdim\@tempa\p@<\p@\scalebox{\@tempa}{\usebox\pandoc@box}%
  \else\usebox{\pandoc@box}%
  \fi%
}
% Set default figure placement to htbp
\def\fps@figure{htbp}
\makeatother

\makeatletter
\@ifpackageloaded{caption}{}{\usepackage{caption}}
\AtBeginDocument{%
\ifdefined\contentsname
  \renewcommand*\contentsname{Table of contents}
\else
  \newcommand\contentsname{Table of contents}
\fi
\ifdefined\listfigurename
  \renewcommand*\listfigurename{List of Figures}
\else
  \newcommand\listfigurename{List of Figures}
\fi
\ifdefined\listtablename
  \renewcommand*\listtablename{List of Tables}
\else
  \newcommand\listtablename{List of Tables}
\fi
\ifdefined\figurename
  \renewcommand*\figurename{Figure}
\else
  \newcommand\figurename{Figure}
\fi
\ifdefined\tablename
  \renewcommand*\tablename{Table}
\else
  \newcommand\tablename{Table}
\fi
}
\@ifpackageloaded{float}{}{\usepackage{float}}
\floatstyle{ruled}
\@ifundefined{c@chapter}{\newfloat{codelisting}{h}{lop}}{\newfloat{codelisting}{h}{lop}[chapter]}
\floatname{codelisting}{Listing}
\newcommand*\listoflistings{\listof{codelisting}{List of Listings}}
\makeatother
\makeatletter
\makeatother
\makeatletter
\@ifpackageloaded{caption}{}{\usepackage{caption}}
\@ifpackageloaded{subcaption}{}{\usepackage{subcaption}}
\makeatother

\usepackage[style=authoryear-comp,]{biblatex}
\addbibresource{references.bib}
\usepackage{bookmark}

\IfFileExists{xurl.sty}{\usepackage{xurl}}{} % add URL line breaks if available
\urlstyle{same} % disable monospaced font for URLs
\hypersetup{
  pdftitle={Comments on: Exploratory functional data analysis},
  pdfauthor={Rob J Hyndman},
  colorlinks=true,
  linkcolor={blue},
  filecolor={Maroon},
  citecolor={Blue},
  urlcolor={Blue},
  pdfcreator={LaTeX via pandoc}}

% Fonts
\usepackage{bera}
\usepackage[charter]{mathdesign}
\usepackage[lf,t]{FiraSans}
\usepackage[scale=0.9]{sourcecodepro}

%% Line and page breaking
\sloppy
\raggedbottom
\usepackage[bottom]{footmisc}
\clubpenalty = 10000
\widowpenalty = 10000
\brokenpenalty = 10000
\allowdisplaybreaks
\usepackage{microtype}

% Captions
\usepackage{caption}
\DeclareCaptionStyle{italic}[justification=centering]
 {labelfont={bf},textfont={it},labelsep=colon}
\captionsetup[figure]{style=italic,format=hang,singlelinecheck=true}
\captionsetup[table]{style=italic,format=hang,singlelinecheck=true}

%% Float placement
\setcounter{topnumber}{2}
\setcounter{bottomnumber}{2}
\setcounter{totalnumber}{4}
\renewcommand{\topfraction}{0.85}
\renewcommand{\bottomfraction}{0.85}
\renewcommand{\textfraction}{0.15}
\renewcommand{\floatpagefraction}{0.8}

% Section titles
\usepackage[compact,sf,bf]{titlesec}
\titleformat{\section}[block]
  {\fontsize{15}{17}\bfseries\sffamily}
  {\thesection}
  {0.4em}{}
\titleformat{\subsection}[block]
  {\fontsize{12}{14}\bfseries\sffamily}
  {\thesubsection}
  {0.4em}{}
\titlespacing{\section}{0pt}{*5}{*1}
\titlespacing{\subsection}{0pt}{*2}{*0.2}
\titlespacing{\subsubsection}{0pt}{*1}{*0.1}

%% Headers and footers
\usepackage{fancyhdr}
\pagestyle{fancy}
\lfoot{}\cfoot{}\rfoot{}
\lhead{\textsf{Comments on: Exploratory functional data analysis}}
\rhead{\textsf{\thepage}}
\setlength{\headheight}{15pt}
\renewcommand{\headrulewidth}{0.4pt}
\fancypagestyle{plain}{%
\fancyhf{} % clear all header and footer fields
\fancyfoot[C]{\sffamily\thepage} % except the center
\renewcommand{\headrulewidth}{0pt}
\renewcommand{\footrulewidth}{0pt}}

%% BIBLIOGRAPHY.

\makeatletter
\@ifpackageloaded{biblatex}{
\ExecuteBibliographyOptions{bibencoding=utf8,minnames=1,maxnames=3, maxbibnames=99,dashed=false,terseinits=true,giveninits=true,uniquename=false,uniquelist=false,doi=false, isbn=false,url=true,sortcites=false}
\DeclareFieldFormat{url}{\texttt{\url{#1}}}
\DeclareFieldFormat[article]{pages}{#1}
\DeclareFieldFormat[inproceedings]{pages}{\lowercase{pp.}#1}
\DeclareFieldFormat[incollection]{pages}{\lowercase{pp.}#1}
\DeclareFieldFormat[article]{volume}{\mkbibbold{#1}}
\DeclareFieldFormat[article]{number}{\mkbibparens{#1}}
\DeclareFieldFormat[article]{title}{\MakeCapital{#1}}
\DeclareFieldFormat[article]{url}{}
\DeclareFieldFormat[inproceedings]{title}{#1}
\DeclareFieldFormat{shorthandwidth}{#1}
\usepackage{xpatch}
\xpatchbibmacro{volume+number+eid}{\setunit*{\adddot}}{}{}{}
% Remove In: for an article.
\renewbibmacro{in:}{%
  \ifentrytype{article}{}{%
  \printtext{\bibstring{in}\intitlepunct}}}
\AtEveryBibitem{\clearfield{month}}
\AtEveryCitekey{\clearfield{month}}
\DeclareDelimFormat[cbx@textcite]{nameyeardelim}{\addspace}
\renewcommand*{\finalnamedelim}{\addspace\&\space}
}{}
\makeatother


%%% Change title format and allow branding
\usepackage{color,titling,framed}


\pretitle{%

\vspace*{-1.1cm}
\LARGE\bfseries}
\posttitle{\vspace*{0.3cm}\par}
\preauthor{\large}
\postauthor{\hfill}
\predate{\small}
\postdate{\vspace*{0.cm}}

\let\oldmaketitle\maketitle
\def\maketitle{
\vspace*{-2cm}
\definecolor{shadecolor}{RGB}{210,210,210}
\begin{snugshade}\sffamily
\oldmaketitle
\end{snugshade}\vspace*{0.5cm}
\definecolor{shadecolor}{RGB}{248,248,248}
}

\title{Comments on: Exploratory functional data analysis}
\author{Rob J Hyndman}
\date{16 January 2025}

\begin{document}
\maketitle
\begin{abstract}
A useful approach to exploratory functional data analysis is to work in
the lower-dimensional principal component space rather than in the
original functional data space. I demonstrate this approach by finding
anomalies in age-specific US mortality rates between 1933 and 2022. The
same approach can be employed for many other standard data analysis
tasks, and has the advantage that it allows immediate use of the vast
array of multivariate data analysis tools that already exist, rather
than having to develop new tools for functional data.
\end{abstract}


\setstretch{1.3}
\textcite{efda} have produced a fascinating paper on the tools that are
available for exploratory analysis of functional data. Much of the
literature has focused on statistical models for functional data, and
related theory, so it is great to see the important pre-modelling work
receiving some attention.

Amongst the methods they describe, several use functional principal
component decomposition \autocite{RD91} to transform the functional data
into a lower-dimensional space. Then some standard EDA tools are applied
to the first few principal component scores, and the results translated
back into the original functional space. For example, this was the
approach used in the functional bagplot and functional HDR boxplot
proposed in \textcite{HS10}. While there is no guarantee that the
features of interest that are present in the original functional data
will be preserved in the PCA space, in practice this almost always leads
to useful results. As well as providing some helpful visualization
tools, this approach can also be used for anomaly detection, giving an
alternative approach to those methods based on statistical depth that
are discussed by \textcite{efda}.

Figure~\ref{fig-us-mortality} (left) shows US mortality rates between
1933 and 2022, obtained from the \textcite{HMD}. Each line denotes the
mortality rates as a function of age for one year, with the colors in
rainbow order corresponding to the years of observation. Overall, we see
a large decrease in mortality rates during early childhood years, then
an increase during teenage years. After about age 30, the rates increase
almost linearly on a log scale. Comparing the curves over time, we see
that the rates have steadily fallen for all ages up to about 95 years,
with more than a 10-fold reduction in mortality rates at around age 10.

\begin{figure}[!htb]

\centering{

\pandocbounded{\includegraphics[keepaspectratio]{EFDA_comments_files/figure-pdf/fig-us-mortality-1.pdf}}

}

\caption{\label{fig-us-mortality}Left: US age-specific mortality rates
for 1933--2022. Right: Annual differences in log mortality rates.}

\end{figure}%

\begin{figure}[!htb]

\centering{

\includegraphics[width=0.79\linewidth,height=\textheight,keepaspectratio]{EFDA_comments_files/figure-pdf/fig-pca-scores-1.pdf}

}

\caption{\label{fig-pca-scores}Pairwise scatterplots of the first three
principal component scores for the US age-specific log mortality annual
differences.}

\end{figure}%

Clearly the data are non-stationary due to the steady decline over time,
so we consider the differences in the log mortality rates over time, as
shown in right panel. Now several functional observations stand out as
having different behaviour from the others, including three (in purple)
from the last few years of data.

We will use principal component scores to detect anomalies in this data
set. Because we are interested in anomaly detection, we do not want the
principal component decomposition to be affected by the anomalies we are
trying to detect. Consequently, we will use the robust principal
component method proposed by \textcite{pprpca}, applied to the annual
differences in the log mortality rates (shown on the right of
Figure~\ref{fig-us-mortality}), to obtain the first three principal
component scores. These are shown in Figure~\ref{fig-pca-scores}. The
loadings (not shown) suggest that the first PC corresponds to ages
0--40, the second PC increases with age after age 30, while the third PC
contrasts children under 10 with people above age 25.

The lookout anomaly detection algorithm \autocite{lookout2021} has been
applied to the first three PC scores. This estimates a multivariate
kernel density estimate of the 3-dimensional data set, and fits a
generalized Pareto distribution (GPD) to the top 10\% of the most
extreme ``surprisal'' values (equal to minus the log of the estimated
density at each observation). Those points with probability less than
0.5 under the GPD are labelled in Figure~\ref{fig-pca-scores} (giving an
effect false positive rate of 5\%). The last three years of data
(2020--2022) are identified as anomalies (probably due to COVID-19),
along with 1936 (at the end of the Great Depression,
\textcite{tapia2009life}) and 1947 (due to rapid improvement in
mortality after WW2). Note that all three principal component scores are
needed to identify these anomalies. War deaths are excluded from the
data set, as they took place outside the country, so the war years are
not seen as anomalous.

Figure~\ref{fig-outliers} shows the years identified as anomalous
against the backdrop of all other years in the data set. While the last
three years stand out from the rest of the data, the data for 1936 and
1947 are not so obviously anomalous from the data plot alone.

\begin{figure}[!htb]

\centering{

\pandocbounded{\includegraphics[keepaspectratio]{EFDA_comments_files/figure-pdf/fig-outliers-1.pdf}}

}

\caption{\label{fig-outliers}}

\end{figure}%

When the ``directional outlyingness'' method of
\textcite{dai2019directional} is applied to these data, only 1947, 2020,
2021 and 2022 are identified as anomalies. The increase in mortality at
the end of the Great Depression is missed, although it is arguably more
anomalous than 1947 (especially after age 70) which is identified.

This general approach to exploratory functional data analysis, using the
PCA space rather than the original functional data space, can be
employed for many other standard data analysis tasks such as assessment
of data quality, identifying trends and seasonality, change point
detection, density estimation, feature engineering, and more. The
advantage is that it allows immediate use of the vast array of
multivariate data analysis tools that already exist, rather than having
to develop new tools for functional data. It provides a familiar and
computationally efficient set of tools that is complementary to those
that work more directly in the functional data space.

The code to reproduce the results in these comments is available at
\url{https://github.com/robjhyndman/EFDA}.


\printbibliography



\end{document}
